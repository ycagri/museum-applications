\subsection{Pargalı İbrahim Paşa Sarayı İç Avlusu}
\indent\indent Oldukça geniş bir alan üzerine inla edilmiş olan İbrahim Paşa Sarayı'nın gümüzde sadece 2/5 ayaktadır. Dört avlu, ahırlar, bir kule ve hazine binasından oluşan saray, oldukça geniş bir araziye yayılmış durumdaydı. Sarayın üçüncü avlusuna 1910 yılında Mimar Vedat Tek'in projelendirdiği Defter-i Hakani(Tapu Kadastro Müdürlüğü)(günümüzde Ayasofya Müzesi) binası yapılmıştır. Sarayın bir diğer avlusu ise 1939 yılında Adliye Sarayı'nın inşası için yıkılmıştır. 1970lerde restorasyon geçiren sarayın ikinci avlusu, 1983 yılında Türk İslâm Eserleri Müzesi'ne tahsis edilmiştir. Saraya gelen misafirleri karşılamak için kullanılan müze avlusunun batı duvarının arkasında ise, İbrahim Paşa'nın kendi hususi avlusu bulunmaktadır. Dikdörtgen forma sahip bu avludan, iki katlı şekilde inşa edilmiş şekilde sarayın ikametgâh kısmı bulunmaktadır.