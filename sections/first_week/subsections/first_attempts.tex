\subsection{Osmanlı Devleti'nde İlk Müzecilik Girişimleri}
\indent\indent Rönesans ile Avrupa'da Yunan ve Roma uygarlıklarına olan ilgi arttı. Bunun sonucunda pek çok gezgin antik Yunan ve Roma şehirlerini ziyaret etmeye başlamıştır. 18.yüzyılın ikinci yarısından sonra bu ilgi Mısır'ı da içine almaya başladı. Bu ilginin sebebini, Avrupalılar'ın Osmanlı toprakları üzerinde hak sahibi olma kaygısından dolayı olduğu ileri sürülmüştür.\cite{shaw} Bunun sonucunda 19.yüzyılın başında Osmanlı topraklarında yabancılar tarafından kazılar yapılmaya başlandı. Bu dönemde Osmanlı Devleti'nde eski eser kavramı tam olarak yoktu. Eski eserlerin hukuki durumunu fıkıh belirlemiştir. Hangi tür arazide bulunursa bulunsun, sahibi belli olmayan eşyalar için aşağıdaki kurallar uygulanır.\cite{mumcu_1}
\begin{enumerate}[label=\alph*)]
    \item üstü kelime-i şehadet ya da İslam için tanınmış başka bir işaret ile süslü ise bu eşyalar lukata(bulunutu mal) hükmündedir. Lukata hükümlerine göre, eşyayı bulan, malın sahibinin ortaya çıkması için durumu beyan eder. Sonuç alınazmsa bulan kişinin ekonomik durumuna bakılır. Yoksul ise kendisi alır, zengin ise ya fakirlere ya da Beytülmal'a alır.
    \item üzerinde İslamdan başka dinlere ait işaretler ya da İslam olmayan hükümdarların adları kazınmış eşyaların beşte biri Beytülmal'a alınır. Geriye kalanı, arazi padişah tarafından kime tahsis edildiyse ona ya da mirasçılarına verilir. Eğer arazi kimseye tahsis edilmemişse ve miri de değilse, eşyanın geri kalanı bulana verili. Bulan kişi Osmanlı Devleti vatandaşı değilse bu haktan faydalanamaz. Yalnızca padişah izni ile define arayanlar, kendisine verilen hisseyi alabilir.
    \item Bulunan eşyanın kategorisi belirlenemiyorsa, (b) bendindeki hükümler uygulanır.
\end{enumerate}
\indent\indent Buradan anlaşılıyor ki, kazı sonucu bulunan eşyalara el koymak oldukça kolaydır. Bu kayıtsızlığın sebebini belki de Osmanlı Devleti'nin eski eser tanımında aramak gerekir. Fıkhın belirlediği maddelerden de anlaşılacağı gibi, Osmanlı Devleti Türk-İslam kültüründe değerli olan eşyaları eser olarak görüyordu. Büyük bir imparatorluk olduğu için, kendi içinden çıkan değerlere önem veriyordu. Bu noktada antik Yunan ve Roma medeneyitlerine ait kalıntılara önem verdiğini söyleyemeyiz. Bundan dolayıdır ki, 19.yüzyılda padişahtan kazı izni almak oldukça kolaydı. Bu izinler, 1840'larda Maarif Nezareti'nin yetkisine verilmiştir.\cite{dilbaz_1}\newline
\indent Osmanlı Devleti'nde ilk müze girişimi, eski eserlerin \textit{Mecma-ı Eslah-ı Atik}(Eski Silahlar Koleksiyonu) ve \textit{Mecma-ı Âsâr-ı Atîka}(Eski Eserler Koleksiyonu) adı altında, o dönem silah deposu olarak kullanılan  Aya İrini'de  Tophane Müşiri Fethi Ahmet Paşa tarafından toplanması ortaya çıkmıştır.\cite{degirmenci} Bu girişime tam olarak bir müze demek doğru olmayacaktır; çünkü eserlerin toplanma amacı sergilemesinden çok koruma altına alınmasıydı. Fethi Ahmet Paşa, 1858'deki ölümüne kadar eserlerin Aya İrini'de toplanması görevini yürüttü. Fethi Ahmet Paşa'nın ölümünden 11 yıl sonra Aya İrini, Maarif Nazırı Saffet Paşa tarafından \textit{Müze-i Humay\^{u}n} adı altında ziyarete açıldı. Maarif Nazırı Saffet Paşa'nın diğer bir önemli icraatı ise, vilayetlere gönderdiği yazılar ile ortaya çıkan eski eserlerin Müze-i Humay\^{u}n'a gönderlimesini buyurmasıydı. Bunun sonucunda, kısa bir süre sonra Aya İrini toplanan eski eserler için yetersiz kalmaya başladı. Yeni müze binası olarak Topkapı Sarayı'ndaki Çinili Köşk tayin edildi. Eserlerin Çinili Köşk'e taşınması 1881 yılında tamamlanmış ve yeni Müze-i Humay\^{u}n törenle açılmıştır. Burası, bugünkü İstanbul Arkeoloji Müzesi'nin çekirdeğini oluşturacaktır.
\indent Müze-i Humayûn'un ilk müdürü olarak Mekteb-i Sultânî tarih öğretmenlerinden  Edward Goold(1868-1871) atandı. Müzenin ikinci müdürü, bu görevi yaklaşık bir yıl kadar sürdürecek olan Pio Francesco Carlo Terenzio'dur. Terenzio'dan boşalan koltuğa 1872 yılında Philipp Anton Dethier oturmuş ve bu görevi 1881 yılına kadar sürdürmştür.\cite{saatci}