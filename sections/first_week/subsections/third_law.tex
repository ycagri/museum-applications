\subsection{1884 Asar-ı Atika Nizamnamesi}
\indent\indent Osman Hamdi Bey, Müze-i Hümayûn'un müdürlüğü görevini yürütürken 1864 nizamnamesinin yetersizliğini fark etti. Girişimleri ile 1884'te üçüncü bir nizamname yayınladı. Beş bölüm ve 37 maddeden oluşan nizamname 1906 yılında gözden geçirilip tekrar yayınlandı ve 1973 yılına kadar büyük bir değişiklik geçirmeden yürürlükte kaldı.\cite{dilbaz_2} Nizamnameyi şu şekilde özetleyebiliriz:\cite{mumcu_3}
\begin{itemize}
    \item 1.maddede, 1874 nizamnamesinde olduğu gibi eski eser tanımı üzerinde durulmuş ve daha kapsamlı hale getirilmiştir.
    \item 2.maddede eski eserlerin temellük ve tassaruf hakları belirtilmiştir.
    \item 3.maddeden 6.maddeye kadar eski eserlerin korunması gerekliliği, yıkılıp kaldırılamayacağı ya da yapılar üzerinde herhangi bir değişiklik yapılamayacağı karara bağlanmıştır.
    \item 8.madde yurt dışına eser ihracını kesin ve istinisnasız olarak yasaklamıştır.
    \item 9.maddeden 27.maddeye kadara kazı için alınması gerek ruhsatnamenin şartları çizilmiştir.
    \item 33.maddeden 35.maddeye kadar eski eser hukukun uygulayacağı cezaları açıklamıştır.
    \item 36.maddede, ortaya çıkan anlaşmazlıkların adli mahkemeler tarafından çözüleceği hükme bağlanmıştır.
\end{itemize}