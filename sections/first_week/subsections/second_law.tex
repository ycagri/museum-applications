\subsection{1874 Asar-ı Atika Nizamnamesi}
\indent\indent 1869 çıkarlan nizamname, eski eserler hukuku üzerine sadece bir başlangıç oldu. Kısa sürede bu ilk nizamnamenin yetersizliği anlaşıldığı için 1874 yılında daha detaylı bir şekilde, ikinci bir nizamname yayınlandı. 34 maddeden oluşan bu nizamnameyi şu şekilde özetleyebiliriz.\cite{mumcu_2}
\begin{itemize}
    \item İlk iki madde eski eser tanımını yapmaktadır.
    \item Sonraki dört madde esas ilkeleri açıklamaktadır. Keşfedilmemiş eski eser nerede bulunursa bulunsun, devlete aittir prensibi kabul ediliyor. İzin alarak yasal kazı yapanların buluntunun üçte birini devlete, diğer üçte birini arazi sahibine, geriye kalan üçte biri de kendisine alabileceğini belirtiyor.
    \item Sonraki maddede, önceki nizamnamede olduğu gibi, kazıları Maarif Nezareti'nin iznine bağlıyor.
    \item 8.maddeden 22.maddeye kadar kazı izni için gerekli olan şartları ve ödenmesi gereken harçları belirliyor.
    \item 23.maddede, sahipsiz alanda devlet kazı yapmak isterse başka kimseye o alanda kazı izni verilmeyeceği yasasını koyuyor.
    \item 24.maddede, devletin özel mülklerde de kazı yapabileceğini ve özel mülkte oluşan hasarı tazmin edeceğini ekliyor.
    \item Eski eser bulanların en geç 10 gün içinde mahalli devlet otoritelerine başvurması gerektiği, eğer bu yasa ihlal edilirse şahıslara devletin payının dörtte biri kadar para cezası uygulanması 25.maddede açıklanıyor.
    \item 28. maddede, buluntunun nasıl bölüştürüleceği çözülemezse, Maarif Nezareti'nin başvurulması gerektiği belirtiliyor.
    \item 31.maddeden 34.maddeye kadar, eski eserlerin satışı, ihracı ve ithalıyla ilgili yasaları ortaya koyuyor.
\end{itemize}