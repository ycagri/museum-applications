\subsection{1869 Asar-ı Atika Nizamnamesi}
\indent\indent 1869 yılına kadar Osmanlı Devleti'nde eski eserlerin durumunu fıkıh dışında belirleyen bir kanun ya da nizamname bulunmamaktaydı. 1869 yılında, Sadrazam Ali Paşa'nın sadareti dönemimde eski eserlerle ilgili ilk nizamname yayınlanmıştır. Bu nizamnamenin giriş kısmında birtakım tespitlere ve hususlara yer verilmiş, Maarif Nezareti'nin girişimiyle kaleme alındığı belirtilip, maddelere geçilmiştir. Tespitler, hususları ve maddeler aşağıdaki gibi özetlenebilir:\cite{karaduman}
\subsubsection{Tespitler}
\begin{itemize}
    \item Eski eserlerin Osmanlı Devleti topraklarında çokça bulunması
    \item Kazılar için ruhsat verilirken çift olarak bulunan eski eserlerin bir tekinin Devlet-i Aliyye'ye verilmesi usulunun, ikili eserlerin pek nadir çıkmasından ya da ikili çıkanlarının bir eşinin saklanmasından dolayı uygulanamaması
\end{itemize}
\subsubsection{Hususlar}
\begin{itemize}
    \item Eski eserlerin kıymet ve öneminden dolayı yapılacak araştırma ve kazıların yazılı kurallara bağlanması
    \item Müzenin düzenlenerek mevcut eserlerin kayıt altına alınması, sergilenmesi ve eksiklerinin giderilmesi hususlarının Maarif Nezareti'ne bağlanması
    \item Müzenin masraflarının karşılanması için adı geçen bakanlığa ücret tahsisi
\end{itemize}
\subsubsection{Maddeler}
\begin{itemize}
    \item Eski eser kazı ve araştırması yapacak kişiler, Maarif Nezareti'nden izin almadıkça hiçbir kazı yapamayacaktır.
    \item Kazı ve araştırma izni alan kişilerin buldukları eski eserleri yurt dışına çıkaramayacaktır. İçeride talep olursa, bu eserleri Devlet-i Aliyye'ye satma hakları mevcuttur.
    \item Özel mülkte çıkan eski eserler mülk sahibinin olacaktır.
    \item Bulunan sikkeler, yurt dışına çıkarma yasağından muaftır.
    \item Eski eser araştırması izni yalnızca toprak altını kapsamaktadır.
    \item Bir devletin resmi olarak eski eser talebinde bulunması halinde, bu talebin kabul edilip yerine getirilmesi Padişah'ın özel iznine bağlıdır.
    \item Eski eser araştırma ve kazılarında uzman olan kişilere, masrafları ve ücretleri hazine tarafından karşılanmak üzere memuriyet ve resmi izin verilecektir. Bu nedenle, söz konusu kişiler ilgili nezarete başvuruda bulunacaktır.
\end{itemize}