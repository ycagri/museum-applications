\subsection{Büyük Mabeyn Köşkü}
\indent\indent Osmanlı İmparatorluğu döneminde Yıldız Tepesi üzerinde, Osmanlı saray mimarisinin son dönemine ait Barok, Art Nouveau, Neoklasik üsluplarda çeşitli yapılar inşa edilmiştir. Kanuni Sultan Süleyman devrinden itibaren padişahlar tarafından av sahası olarak kullanılan ve Hazine-i Hassa’ya kayıtlı olan bu arazide ilk yapılaşma, Sultan I. Ahmed (1603–1617) döneminde gerçekleştirilmiş, padişah burada bir kasır inşa ettirmiştir. 18. yüzyılın sonlarına doğru Sultan III. Selim (1789–1807), annesi Mihrişah Valide Sultan için bir kasır, babası Sultan III. Mustafa adına ise günümüze ulaşan dört cepheli rokoko tarzında bir çeşme yaptırmıştır. Bu gelişmeleri takip eden süreçte, çeşitli padişahlar bölgede köşk ve kasırlar inşa ettirmeye devam etmişlerdir.\newline
\indent Yıldız Sarayı kompleksinin önemli yapılarından biri olan Büyük Mabeyn Köşkü, 1866 yılında Sultan Abdülaziz tarafından ünlü Ermeni asıllı mimar Sarkis Balyan’a yaptırılmıştır. II. Abdülhamid döneminde bu yapı, resmî toplantıların yanı sıra diplomatik kabul ve davetlerin gerçekleştirildiği mekân haline gelmiştir. Bu bağlamda, 1884 yılında Avusturya-Macaristan Veliaht Prensi Rudolf ve eşi, 1889 yılında ise Alman İmparatoru II. Wilhelm, II. Abdülhamid tarafından Mabeyn Köşkü’nde ağırlanmıştır. Günümüzde bu tarihî yapı, Cumhurbaşkanlığı tarafından kabul sarayı olarak kullanılmaktadır.\cite{dia_7}