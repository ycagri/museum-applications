\subsection{Yıldız Sarayı}
\indent\indent İstanbul Boğazı ve Marmara Denizi’ne hâkim yüksek bir konuma inşa edilen Yıldız Sarayı, Osmanlı İmparatorluğu'nun İstanbul’da yaptırdığı son büyük saray kompleksidir. Mimari yerleşimi bakımından doğuda saray(devlet işlerinin yürütüldüğü bölümler ve harem), batıda ise Yıldız Hamidiye Camii yer almakta olup bu yapı düzeni, geleneksel Türk-İslam saray mimarisini sürdürmektedir. Toplam 500.000 metrekarelik bir alana yayılan saray; yönetim binalarının bulunduğu merkezî saray bölümü, padişah ve haremine ait özel alanlar ve dış bahçe ile çevresindeki yapılardan oluşur. Günümüzde Yıldız Parkı olarak bilinen dış bahçe, Osmanlı döneminde yalnızca saraya aitken, bugün halka açıktır. Bu alan içerisinde geçmişte çok sayıda köşk ve yapı yer almakta olup günümüze yalnızca Çadır, Malta ve Şale köşkleri ulaşmıştır.\newline
\indent Yıldız Sarayı bölgesine inşa edilen ilk köşk, Sultan III. Selim döneminde (1789–1807) yapılmıştır. Her ne kadar sonraki padişahlar da bu bölgeyle ilgilenmişse de Yıldız Sarayı esas olarak Sultan II. Abdülhamid(1876–1909) ile özdeşleşmiştir. II. Abdülhamid, burayı \textit{Yıldız Saray-ı Hümâyûnu} olarak adlandırmış ve 33 yıl boyunca devletin yönetim merkezi olarak kullanmıştır. Beşiktaş ile Ortaköy arasında geniş bir alana yayılan saray, yalnızca köşklerden oluşan bir yapı grubu değil; aynı zamanda bahçeler, havuzlar, seralar, tiyatro, müze, kütüphane, atölyeler, çini fabrikası gibi birçok işlevsel ve sanatsal birimi barındıran çok yönlü bir komplekstir. Ayrıca padişah, I. Ordu’ya bağlı II. Tümen’i(Orhaniye ve Ertuğrul kışlaları) saray çevresine konuşlandırmıştır. Döneminde saray çevresinde doğrudan ya da dolaylı biçimde çalışanlarla birlikte yaklaşık 12.000 kişilik bir nüfusun bulunduğu tahmin edilmektedir. Yıldız Sarayı, Osmanlı'nın son yönetim merkezi ve padişahın son resmî ikametgâhı olarak tarihsel bir önem taşımaktadır.\newline
\indent Sarayın girişi kısmına inşa ettirilmiş olan Yıldız Camii'nin de tarihti önemli bir yeri vardır. 21 Temmuz 1905 tarihinde Ermeni Devrimci Federasyonu(Taşnaksutyun) tarafından II. Abdülhamid'e yönelik suikast girişimi, Yıldız Camii önünde düzenlenmiştir. Suikast planına göre, II. Abdülhamid’in cuma selamlığından çıkışı sırasında cami önüne park edilmiş bir bomba yüklü araba infilak ettirilerek padişah öldürülmek istenmiştir. Ancak Abdülhamid, her zamanki alışkanlığıyla camiden çıkışta cemaatle kısa süre sohbet ettiğinden, patlama padişah camiden çıkmadan önce gerçekleşmiş ve başarısız olmuştur. Saldırıda 26 kişi ölmüş, çok sayıda kişi de yaralanmıştır; ancak II. Abdülhamid yara almadan kurtulmuştur. Suikastın arkasındaki isimlerden Belçikalı anarşist Edward Joris tutuklanmış, olay uluslararası boyut kazanmıştır. Suikast girişimi, Osmanlı yönetiminin güvenlik önlemlerini artırmasına yol açmış, II. Abdülhamid’in güvenlik kaygılarını daha da derinleştirmiştir.\newline
\indent Saray yerleşkesi, iç içe geçmiş avlularla birbirinden ayrılmış olan iç saray bölümü (harem ve mâbeyin), Has Bahçe ve dış bahçe (Yıldız Parkı) olmak üzere üç ana kısımdan oluşur. II. Abdülhamid döneminde saray, Osmanlı yönetim merkezine dönüşmüş ve tiyatrodan marangozhaneye, çini atölyesinden kütüphaneye kadar çok yönlü yapılarla donatılmıştır.