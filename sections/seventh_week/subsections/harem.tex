\subsection{Harem}
\indent\indent Sarayın mahrem alanını oluşturan harem, Vâlide Sultan Köşkü çevresinde şekillenmiştir. Abdülmecid döneminde inşa edilen ve II. Abdülhamid devrinde genişletilen bu köşk, bağdâdî teknikle yapılmış iki katlı yapısı, geniş salonları ve kalem işi süslemeleriyle dikkat çeker. Köşkün karşısında yer alan harem yapıları; musâhipler, kızlarağası, kadınefendiler, hazinedarlar ve câriyelere ait dairelerden oluşur. Tüm yapılar hiyerarşik bir düzende konumlandırılmış ve estetik bütünlük içerisinde tasarlanmıştır. Harem kısmına bağlı olarak inşa edilen Yıldız Sarayı Tiyatrosu, 1889’da tamamlanmış ve hem Batı tiyatrosu hem de geleneksel Türk sahne sanatlarına ev sahipliği yaparak harem kadınlarının kültürel yaşamında önemli bir yere sahip olmuştur.\cite{dia_7}