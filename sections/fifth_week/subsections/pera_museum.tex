\subsection{Müzeye Genel Bakış}
\indent\indent 1883 yılında \textit{Orient Express}(Doğu Ekspresi)'nin hizmete sunulmasıyla, Pera bölgesinde Avrupa tarzı otelcilik hizmetleri oluşmaya başladı. Dönemin İstanbul'unda han tarzı konaklama servisleri, Doğu Ekspresi'nin lüksüne ayak uyduracak seviyede değillerdi. Bu sebeple, bugünkü Meşrutiyet Caddesi üzerinde 1892 yılında Büyük Londra Oteli ve Pera Palace, 1893 yılında da Bristol Otel inşa edildi. Büyük Londra Oteli ve Pera Palace günümüzde de hizmet vermeye devam ederken, Bristol Otel 1980'de kapanıyor ve bina \textit{Esbank}(Eskişehir Bankası) tarafından satın alınıyor. Esbank'ın 2001 yılında kapanmasıyla birlikte, Suna ve İnan Kıraç Vakfı binayı 2003 yılında satın alıyor. Bina, 2003-2005 yılları arasında restoratör mimar Sinan Genim'in hazırladığı proje doğrultusunda restore edildi. Bu restorasyon ile cadde kotunun 8 metre aşağısına inilereke iki kat bodrum ilavesiyle bina 8 kata ulaştı.\newline
\indent En üst kattan aşağı doğru gezilecek şekilde tasarlanan müzenin en üst 3 katı, her yıl müze küratörleri tarafından 3-4 farklı sergiye ev sahipliği yapıyor. Ziyaretin yapıldığı tarihte en üst iki kat Kanadalı sanatçı Marcel Dzama'nın \textit{Dancing with the Moon}(Ay Işığyla Dans) isimli sergisine ayrılmıştı. Üçüncü katta ise, müzenin kuruluşunda büyük katkıları bulunan ve 2007 yılında vefat eden Samih Rıfat'a vefa sergisi bulunmaktaydı. Müzenin ikinci ve birinci katları ise Kıraç ailesinin koleksiyonlarına ayrılmıştır. İkinci katta ailenin resim koleksiyonu, birinci katta ise ağırlık ve Kütahya çinileri koleksiyonları sergilenmektedir. 