\subsection{Fuat Sezgin}
\indent\indent İstanbul İslam Bilim ve Teknoloji Tarihi Müzesi, Fuat Sezgin'in öncülüğünde 25 Mayıs 2008 tarihinde hizmete girmiştir. 23 Ekim 1924'te doğan Fuat Sezgin, İstanbul Üniversitesi Edebiyat Fakültesi Şarkiyat Enstitüsü'nde öğrenim gördü. 1950 yılında Arap Dili ve Edebiyatı bölümünde \textit{Buhari'nin Kaynakları} adlı doktora tezini yayımladı. 27 Mayıs 1960 asker darbesinin ertesinde, askeri darbe yönetiminin üniversitelerden ihraç ettiği ve 147'ler olarak anılan akademisyenler arasındaydı. İhraç sonrası 1961 yılında Almanya'ya giden Fuat Sezgin, Frankfurt'taki Johann Wolfgnag Goethe Üniversitesi'nde misafir doçent olarak dersler vermeye başladı. 1965 yılında da profesör unvanını aldı.\newline
\indent İstanbul'dayken başladığı 7.-14.yüzyıllar arası Arap-İslam edeibyat tarihi üzerine yaptığı çalışmalarının ilk cildini 1967 yılında, son cildini ise 2000 yılında yayımlamıştır. 13 cild şeklinde yayımladığı \textit{Geschichte des arabischen Schrifttums} isimli eser, Arap-İslam kültürünün coğrafya, haritacılık, edebiyat tarihine ışık tutmaktadır.